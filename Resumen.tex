\chapter{Resumen}

\textbf{\Large Rodamientos Magn�ticos H�bridos}\\

\textbf{Palabras Clave:}  Rodamientos, levitaci\'on magn\'etica, electroimanes, control por variable estado, control discreto, sensores, carga radia, carga axial, simulaci\'on magn\'etica, \'areas funcionales, materiales ferromagn\'eticos. 

\subsubsection{Resumen}

En este trabajo se simul� el comportamiento de un rodamiento magn�tico h�brido que implementa en su dise�o un rodamiento magn�tico activo tanto para la carga axial como para la carga radial, as� como algunas caracter�sticas de los rodamientos magn�ticos pasivos para proveer un campo magn�tico base de levitaci�n. Se estudi� el fen�meno de la levitaci�n magn�tica y sus implicaciones te�ricas para definir los m�dulos funcionales que se requirieron en el dise�o. A partir de un control integral calendarizado por variables de estado, se mantiene un control teorizado sobre un eje para una capacidad de carga m�xima est�tica de 50 N, con la particularidad de que el eje puede estar constituido de cualquier material.

\subsubsection{Abstract}

In this paper, the behavior of a hybrid magnetic bearing was simulated, which implements in its design an active magnetic bearing for both the axial load and the radial load, as well as some characteristics of the passive magnetic bearings to provide a levitation base magnetic field. It was studied the phenomenon of magnetic levitation and its theoretical implications to define the functional modules that were required in the design. Based on a scheduled state feedback integral controler, a theorized control is maintained on a shaft for a maximum static load capacity of 50 N, with the particularity that the shaft can be constituted of any material.

