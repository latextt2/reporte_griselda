\chapter{Objetivos}

\section*{Objetivo general}
Dise�ar y simular un sistema de rodamientos magn�ticos capaz de operar en condiciones de carga est�tica.

\section*{Objetivos Particulares.}

{\setlength{\parindent}{0pt}Trabajo Terminal I:}
\begin{enumerate}
\addtolength{\itemsep}{0pt}
\item Obtener el dise�o conceptual del rodamiento magn�tico h�brido.
\item Seleccionar el material ferromagn�tico para el n�cleo de los electroimanes. 
\item Calcular la secci�n transversal de los electroimanes.
\item Obtener la geometr�a y dimensiones de los rodamientos activos radial y axial.
\item Validar el punto de operaci�n del electroim�n mediante simulaci�n.
\item Dise�ar un electrodo de alta sensitividad para el rango de operaci�n.
\item Dise�ar un dispositivo para la retracci�n del im�n permanente. 
\item Proponer un circuito de adquisici�n y control. 
\end{enumerate}
\hfill \break

\newpage

{\setlength{\parindent}{0pt}Trabajo Terminal II:}
\begin{enumerate}
\addtolength{\itemsep}{0pt}
\item Obtener un modelo de electroimanes que integre los efectos del flujo marginal.
\item Realizar las simulaciones de la fuerza magn�tica ejercida por los electroimanes y el im�n permanente.
\item Obtener el modelo din�mico del sistema.  
\item Comparar algunos esquemas de control para la estabilizaci�n del sistema.
\item Dise�o del circuito de potencia de los electroimanes. 
\item Estimar los l�mites de operaci�n del rodamiento con base en las simulaciones.
\end{enumerate}