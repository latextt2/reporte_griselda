\chapter{Conclusiones}
%\todo[inline]{Efectos de control el rotor en 3d}
%%\todo[inline]{Modelar en 3d mejora el desempeño del controlador}
%%\todo[inline]{pero esto requirio un modelo mas complejo y la implemetacion de un control calendarizado}
%%\todo[inline]{Efectos del en sistema hibrido}
%%\todo[inline]{El uso del iman permanente permitio reducir de manera considerable el consumo del rodamiento, inluso ante cargas de baja frecencia}

%%\todo[inline]{Aplicaciones futuras}
%%\todo[inline]{El software de simulacion y la teoria desarroda permitira el rapido desarrolo de electroimanes}
%%\todo[inline]{El sensor de capacitancia tiene aplicaciones por separado}
%%\todo[inline]{Los cualidadesdel sistema hibrido pueden exterse a futuros projectos donde el bajo consumo sea importante, ejemplo un riel de levitacion magnetica para el transporte de carga}

