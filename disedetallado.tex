\chapter{Dise\~no Detallado}
\section{Electroimanes}
\todo[inline]{Aplicar la formula considerando solo la reluctancia del gap resulta en}
\todo[inline]{Simular los resultados resulta en}
\todo[inline]{Para el electriman radial la diferencia entre los resultados se debe a los efectos de la reluctacia del nucleo}
\todo[inline]{Para el electroiman radial la diferencia se debe a a los efectos del flujo marignal, debido a la geometria}
\todo[inline]{Integrando ambos efectos obtenemos el modelo}
\todo[inline]{Con el modelo se propone metodo de dise�o que consiste en pasos}
\todo[inline]{Para la asistencia en los calculos se hace el software para el dise�o de electrimanes}
\subsection{Electroiman Radial}
\todo[inline]{Teniendo las condiciones de operacion, las cuales se sacan a partir de las condiciones de los papers}
\todo[inline]{Usando la herramienta arrojan los resultados siguientes}
\todo[inline]{Simulando en femm obtenemos los resultados}
\todo[inline]{Las dimensiones finales del subsistema son ?? son}
\todo[inline]{Los resultados estan dentro de los parametors aceptables, y las simulacion y los calculos del software conduerdan}
\subsection{Electroiman Axial}
\todo[inline]{Teniendo las condiciones de operacion, las cuales se sacan a partir de las condiciones de los papers}
\todo[inline]{Usando la herramienta arrojan los resultados siguientes}
\todo[inline]{Simulando en femm obtenemos los resultados}
\todo[inline]{Las dimensiones finales del subsistema son ?? son}
\todo[inline]{Los resultados estan dentro de los parametors aceptables, y las simulacion y los calculos del software conduerdan}
\subsection{Accionamiento de los electroimanes}
\todo[inline]{Circuito de potencia}
\todo[inline]{Modo lineal vs modo conmutado}
\todo[inline]{El puente H y sus detalles}
\todo[inline]{Rectificaion sincrona}
\todo[inline]{Control de corriente}
\todo[inline]{Circuito final}
\todo[inline]{Calculo de la eficiencia}
\todo[inline]{Simulacion electronica}
\todo[inline]{Respuesta dinamica y los sistemas con limites de slew rate}
\section{M�dulo de Compensaci�n para Cargas Est�ticas}
\todo[inline]{El objetivo del modulo de compsensacion es}
\todo[inline]{El metodo para dise�ar un electrimna pasivo es}
\todo[inline]{Software para el calculo del iman passivo}
\todo[inline]{Simulacion verificacion del iman passivo}
\todo[inline]{Caraterizacion dela fuerza del iman passivo}
\todo[inline]{Modelo del mecanismo}
\todo[inline]{Acionamiento de sistema}
\todo[inline]{Caracteristicas dinamicas}
\section{Sensores Capacitivos de Desplazamiento}
\subsection{Electrodos}
\todo[inline]{Descripcion de los efectos de la geometria en los sosores de capacitancia proyectada}
\todo[inline]{Simulacion femm}
\todo[inline]{Software para calculo de geometria optica}
\todo[inline]{Dimensiones finales del sensore}
\todo[inline]{Caracteristicas finales del sensores}
\subsection{Circuito de acondicionamiento}
\todo[inline]{Conversion de capacitancia a frecuencia}
\section{Integracion}
\todo[inline]{Modelo del eje}
\todo[inline]{Modelo completo del sistema}
\subsection{Controladores}
\todo[inline]{Observador de estados}
\todo[inline]{Estabilzacion}
\todo[inline]{Servocontrol}
\todo[inline]{Discretizacion}
\todo[inline]{Fusion de sensores}
\todo[inline]{El problema del ruido}
\todo[inline]{Integrar los datos de 3 sensores}
\todo[inline]{El aplicando el filtro de kalman}