\chapter{Dise\~no Detallado}
\section{Electroimanes}
Para dise�ar un electriman requerimos de modelo que relacione la geometria del nucleo con la fuerza que este puede ejercer. Sabemos por la ecuacion ?? que la fuerza esta determida por
\begin{equation}
	F =  \frac{(NI)^2}{2}) \frac{d}{dg}(R^{-1})
\end{equation}
por lo que el problema de calcular la fuerza se reduce a calcular la reluctancia del circuto magnetico de los electroimanes.

El electroiman posee una geometria como la de la figura ??, aplicando el analisis de circuitos magneticos el sistema se pude modelar como el circuito de reluctancias de la figura ??. Entonces la reluctancia total del circuito puede escribirse como
\begin{equation}
	R = ???
\end{equation}
Si consideramos los terminos ?? y ?? como insfinificantes entonces la fuerza se puede expresarse como:
\begin{equation}
	F = ???
\end{equation}
Esta ecuacion puede encontrarse en la literatura ??,?? y ??, para llegar a esta ecuacion fue necesario hacer varias suposiciones como despreciar flujo de fluga, el efecto marginal, y que la reluctancia del nucleo es mucho menor a la reluctancia del gap, estas condiciones podrian cuplirse en otros sistemas pero caso de un rodamiento magnetico no.

Si ponemos a prueba  la ecuciones con valores de A,B,y D  resulta en una fuerza de ?? newtons, en cambio si realizamos simulacion en FEM, para el mismo sistema, la simulacion arroja un valor de N.

Si reescribimos la ecuaion ?? pero esta vez sin ignorar los terminos ?? correspondientes a nucleo del electroiman, obtenemos la ecuacion ??
\begin{equation}
	F = ???
\end{equation}
Sustituyendo los valores usados anteriormente, la ecuacion nos arroja una fuerza de ?? newtons. La fuerzas coincide entre la simulacion FEM y las ecuacion que habria sido usada para calcular los electroimanes.


\todo[inline]{Aplicar la formula considerando solo la reluctancia del gap resulta en}
\todo[inline]{Simular los resultados resulta en}
\todo[inline]{Para el electriman radial la diferencia entre los resultados se debe a los efectos de la reluctacia del nucleo}
\todo[inline]{Para el electroiman radial la diferencia se debe a a los efectos del flujo marignal, debido a la geometria}
\todo[inline]{Integrando ambos efectos obtenemos el modelo}
\todo[inline]{Con el modelo se propone metodo de dise�o que consiste en pasos}
\todo[inline]{Para la asistencia en los calculos se hace el software para el dise�o de electrimanes}
\subsection{Electroiman Radial}
\todo[inline]{Teniendo las condiciones de operacion, las cuales se sacan a partir de las condiciones de los papers}
\todo[inline]{Usando la herramienta arrojan los resultados siguientes}
\todo[inline]{Simulando en femm obtenemos los resultados}
\todo[inline]{Las dimensiones finales del subsistema son ?? son}
\todo[inline]{Los resultados estan dentro de los parametors aceptables, y las simulacion y los calculos del software conduerdan}
\subsection{Electroiman Axial}
\todo[inline]{Teniendo las condiciones de operacion, las cuales se sacan a partir de las condiciones de los papers}
\todo[inline]{Usando la herramienta arrojan los resultados siguientes}
\todo[inline]{Simulando en femm obtenemos los resultados}
\todo[inline]{Las dimensiones finales del subsistema son ?? son}
\todo[inline]{Los resultados estan dentro de los parametors aceptables, y las simulacion y los calculos del software conduerdan}
\subsection{Accionamiento de los electroimanes}
\todo[inline]{Circuito de potencia}
\todo[inline]{Modo lineal vs modo conmutado}
\todo[inline]{El puente H y sus detalles}
\todo[inline]{Rectificaion sincrona}
\todo[inline]{Control de corriente}
\todo[inline]{Circuito final}
\todo[inline]{Calculo de la eficiencia}
\todo[inline]{Simulacion electronica}
\todo[inline]{Respuesta dinamica y los sistemas con limites de slew rate}
\section{M�dulo de Compensaci�n para Cargas Est�ticas}
\todo[inline]{El objetivo del modulo de compsensacion es}
\todo[inline]{El metodo para dise�ar un electrimna pasivo es}
\todo[inline]{Software para el calculo del iman passivo}
\todo[inline]{Simulacion verificacion del iman passivo}
\todo[inline]{Caraterizacion dela fuerza del iman passivo}
\todo[inline]{Modelo del mecanismo}
\todo[inline]{Acionamiento de sistema}
\todo[inline]{Caracteristicas dinamicas}
\section{Sensores Capacitivos de Desplazamiento}
\subsection{Electrodos}
\todo[inline]{Descripcion de los efectos de la geometria en los sosores de capacitancia proyectada}
\todo[inline]{Simulacion femm}
\todo[inline]{Software para calculo de geometria optica}
\todo[inline]{Dimensiones finales del sensore}
\todo[inline]{Caracteristicas finales del sensores}
\subsection{Circuito de acondicionamiento}
\todo[inline]{Conversion de capacitancia a frecuencia}
\section{Integracion}
\todo[inline]{Modelo del eje}
\todo[inline]{Modelo completo del sistema}
\subsection{Controladores}
\todo[inline]{Observador de estados}
\todo[inline]{Estabilzacion}
\todo[inline]{Servocontrol}
\todo[inline]{Discretizacion}
\todo[inline]{Fusion de sensores}
\todo[inline]{El problema del ruido}
\todo[inline]{Integrar los datos de 3 sensores}
\todo[inline]{El aplicando el filtro de kalman}