\chapter{Codigos de Matlab}

\section{Par\'ametros del sistema}

\begin{lstlisting}[frame=single]
function [m,R_r,R_a,R_p,I_p,I_0] = system_parameters()
%	UNTITLED3 Summary of this function goes here
%   Detailed explanation goes here

l_shaft = 0.40;
l_radial_to_shaft = 0.052 + 0.020;
l_mcce_to_shaft = 0.052 + 0.060;
l_axial_to_shaft = 0.147;

r_shaft = 12.25e-3;
rho_shaft = 1050; %%plastico ABC
v_shaft = pi*(r_shaft^2)*l_shaft;
m_shaft = v_shaft*rho_shaft;

%%	Datos del collarin
Ixx_collarin = 4588236e-9;
I0_collarin = 6754745e-9;
l_cm_collarin =  87.33e-3;
m_collarin = 3.216;

I_p = (1/2)*m_shaft*(r_shaft^2) + 2*Ixx_collarin;%%Ixx
I_0 = (1/12)*m_shaft*(l_shaft^2) + 2*(I0_collarin + 
		m_collarin*(l_shaft + l_cm_collarin)^2);%%Irho

R_r = l_shaft + l_radial_to_shaft;
R_p = l_shaft + l_mcce_to_shaft;
R_a = l_shaft + l_axial_to_shaft;
m = m_shaft + m_collarin;

end
\end{lstlisting}


\section{Generador de ganancias canlendarizadas}

\begin{lstlisting}[frame=single]
clc
clear all

P = -((1:10)*4 + 300);
Po = -((1:10)*4 + 600); 
Pi = [P,-((1:5)+30)];
Ts = 0.5e-3;
ws_lt = -30000 : 2500 :30000;

generate_non_linear_model();
[xp_nl,y_nl] = get_symbolic_model('control');
Kh = 8;
%%
for n = 1:length(ws_lt)
ws = ws_lt(n);
[A,B,C] = get_linear_model(xp_nl,y_nl,ws);
[K, Ko] = calc_linear_controller_gains(A,B,C,P,Po);
Ki = calc_integral_controller_gains(A,B,C,Pi);

[Ad,Bd,Cd] = get_discrete_linear_model(A,B,C,Ts);
[Kd,Kdo] = calc_discrete_controller_gains(A,B,C,P,Po,Ts);
Kdi = calc_discrete_integral_controller_gains(A,B,C,Pi,Ts);

A_lt(:,:,n) = A; 
B_lt(:,:,n) = B;
C_lt(:,:,n) = C;
K_lt(:,:,n) = K;
Ko_lt(:,:,n) = Ko;
Ki_lt(:,:,n) = Ki;

Ad_lt(:,:,n) = Ad; 
Bd_lt(:,:,n) = Bd;
Cd_lt(:,:,n) = Cd;
Kd_lt(:,:,n) = Kd;
Kdo_lt(:,:,n) = Kdo;
Kdi_lt(:,:,n) = Kdi;
end

ws = 10000;
[A,B,C] = get_linear_model(xp_nl,y_nl,ws);
[K, Ko] = calc_linear_controller_gains(A,B,C,P,Po);
Ki = calc_integral_controller_gains(A,B,C,Pi);

[Ad,Bd,Cd] = get_discrete_linear_model(A,B,C,Ts);
[Kd,Kdo] = calc_discrete_controller_gains(A,B,C,P,Po,Ts);
Kdi = calc_discrete_integral_controller_gains(A,B,C,Pi,Ts);

\end{lstlisting}



\section{Matriz de interpolaci\'on}

\begin{lstlisting}[frame=single]
function [Y_k] = matrix_interpolation(w,Y,w_k)

Y_r = reshape(Y,[size(Y,1)*size(Y,2),size(Y,3)])';
Y_r_k = interp1(w,Y_r,w_k,'linear','extrap');
Y_k = reshape(Y_r_k,[size(Y,1),size(Y,2)]);
end
\end{lstlisting}

\section{Modelo no lineal del RMH} \label{ap:sistema-ecuaciones-no-lineal}

%\newpage
\begin{lstlisting}[frame=single]
function [xp,y] = hmb_non_linear_model
(x,u,ws,Fd_x,Fd_y,Fd_z,Md_y,Md_z,Fmcce1,Fmcce2)

[m,R_r,R_a,R_p,I_p,I_0] = system_parameters();

x1 = x(1);
x2 = x(2);
x3 = x(3);
x4 = x(4);
x5 = x(5);
x6 = x(6);
x7 = x(7);
x8 = x(8);
x9 = x(9);
x10 = x(10);

u1 = u(1);
u2 = u(2);
u3 = u(3);
u4 = u(4);
u5 = u(5);

xp = [
	x4;
	x5;
	x6;
	(Fd_x + u5)/m;
	(Fd_y + u1 + u3)/m;
	(Fd_z + Fmcce1 + Fmcce2 - (49*m)/5 + u2 + u4)/m;
	x9;
	x10/cos(x8);
	-(R_r*(u2*cos(x8) + u1*sin(x7)*sin(x8)) - 
	Md_y - R_r*(u4*cos(x8) + u3*sin(x7)*sin(x8))   	
	+ I_p*ws*x10 + Fmcce1*R_p*cos(x8) - 
	Fmcce2*R_p*cos(x8) + R_a*abs(u5)*cos(x7)*sin(x8))/ I_0;
	(Md_z + R_r*u1*cos(x7) - R_r*u3*cos(x7) - 
	R_a*abs(u5)*sin(x7) + I_p*ws*x9)/I_0;
];

y = [
	x1;
	x2 + R_r*cos(x8)*sin(x7);
	x3 - R_r*sin(x8);
	x2 - R_r*cos(x8)*sin(x7);
	x3 + R_r*sin(x8);
];

end

\end{lstlisting}

\section{Obtenci\'on del modelo}

%\newpage
\begin{lstlisting}[frame=single]
function [xp,y] = get_symbolic_model( model_type )
%	UNTITLED Summary of this function goes here
%   Detailed explanation goes here

%% 	sybolic_constants

syms ws;
syms R_r
syms R_a
syms R_p
syms I_p I_0
syms m

%% state variables
syms x1 x2 x3 x4 x5 x6 x7 x8 x9 x10;
syms u1 u2 u3 u4 u5;

Rcm_x = x1;
Rcm_y = x2;
Rcm_z = x3;
Rcm_p_x = x4;
Rcm_p_y = x5;
Rcm_p_z = x6;

Rcm = [Rcm_x;Rcm_y;Rcm_z];
Rcm_p = [Rcm_p_x;Rcm_p_y;Rcm_p_z];

psi = x7;
theta = x8;
wy = x9;
wz = x10;
wx = ws;
F1_Y = u1;
F1_Z = u2;
F2_Y = u3;
F2_Z = u4;
F3_X = u5;

%% Fuerzas mcce
syms Fmcce1 Fmcce2
F4_Z = Fmcce1;
F5_Z = Fmcce2;

%% Fuerzas pertubadoras
syms Fd_x Fd_y Fd_z Md_y Md_z;

%% Matrices de transformacion al marco movil

T_psi =     [ cos(psi), sin(psi), 0;
            -sin(psi), cos(psi), 0;
            0, 0, 1;];
T_theta =   [ cos(theta), 0, -sin(theta);
            0,1,0;
            sin(theta), 0, cos(theta)];

T_psi_inv = [ cos(psi), -sin(psi), 0;
            sin(psi), cos(psi), 0;
            0, 0, 1;];
T_theta_inv =   [ cos(theta), 0, sin(theta);
                0,1,0;
                -sin(theta), 0, cos(theta)];

T = T_theta*T_psi;
T_inv = T_psi_inv*T_theta_inv;

%% Fuerzas de actuadores
F1 = [0; F1_Y; F1_Z];%%Fuerza RMR derecho
F2 = [0; F2_Y; F2_Z];%%Fuerza RMR izquiero
F3 = [F3_X; 0; 0];%%Fuerza RMA
F4 = [0; 0; F4_Z];%%Fuerza MCCE derecho
F5 = [0; 0; F5_Z];%%Fuerza MCCE izquiero

%% Transformacion de fuerzas al marco movil
F1_xyz = T*F1;
F2_xyz = T*F2;
F3_xyz = T*abs(F3);
F4_xyz = T*F4;
F5_xyz = T*F5;

%% Puntos de accion del actuador respecto al centro del eje
R1 = [R_r;0;0];
R2 = -[R_r;0;0];
R3 = [R_a;0;0];
R4 = [R_p;0;0];
R5 = -[R_p;0;0];

%% Momentos torsores
M1 = cross(R1,F1_xyz);
M2 = cross(R2,F2_xyz);
M3 = cross(R3,F3_xyz);
M4 = cross(R4,F4_xyz);
M5 = cross(R5,F5_xyz);

%%Suma de momentos para control
if(isequal(model_type,'control'))
    M = M1 + M2;
%%Suma de momentos para simulacion
else
    M = M1 + M2 + M3 + M4 + M5 + [0;Md_y;Md_z];
end

%%Matriz de inercia ???
I_m = [I_p, 0, 0;
       0, I_0, 0;
       0 , 0 , I_0];

I_m_inv = inv(I_m);

%% Velocidades angulares
w = [wx;wy;wz];
omega = [wx-ws;wy;wz];

H = I_m * w;

w_p = I_m_inv*(M - cross(omega,H));

%% angle equations
angles_p = [wy;sec(theta)*wz];

%% Suma de fuerzas
if(isequal(model_type,'control'))
%%Suma de fuerzas (para control)
    F = F1 + F2 + F3;
else
%%Suma de fuerzas (para control)
    Fw = 9.8*m * [0; 0; -1];
    F = F1 + F2 + F3 + F4 + F5 + Fw + [Fd_x;Fd_y;Fd_z];
end

%% Ecuaciones dinamicas
Rcm_pp = F/m;

xp = [Rcm_p;
      Rcm_pp;
      angles_p;
      w_p(2:3)];

P1  = T_inv*R1 + Rcm;
P2  = T_inv*R2 + Rcm;

y = [Rcm(1);P1(2:3);P2(2:3)];
end
\end{lstlisting}
 
\section{Modelo lineal} \label{ap:sistema-ecuaciones-lineales}

%\newpage
\begin{lstlisting}[frame=single]
function [Al,Bl,Cl] = get_linear_model(xp_nl,y_nl,ws)
%	UNTITLED4 Summary of this function goes here
%   Detailed explanation goes here
syms x1 x2 x3 x4 x5 x6 x7 x8 x9 x10;
syms u1 u2 u3 u4 u5;

x_state = [x1,x2,x3,x4,x5,x6,x7,x8,x9,x10];
u_state = [u1,u2,u3,u4,u5];

A = jacobian(xp_nl,x_state);
B = jacobian(xp_nl,u_state);
C = jacobian(y_nl,x_state);

for n = 1:length(x_state);
    eval(sprintf('x%d = 0;',n));
end

for k = 1:length(u_state);
    eval(sprintf('u%d = 0;',k));
end

[m,R_r,R_a,R_p,I_p,I_0] = system_parameters();

Al = eval(A);
Bl = eval(B);
Cl = eval(C);
end

\end{lstlisting}
 
\section{Polos discretizados}

%\newpage
\begin{lstlisting}[frame=single]
function [Pd] = get_discrete_poles(P,Ts)
%UNTITLED9 Summary of this function goes here
%   Detailed explanation goes here

my_sys = ss(diag(P),zeros(length(P),1),[1,zeros(1,length(P)-1)],0);
my_sys_d = c2d(my_sys,Ts);
Pd = eig(my_sys_d)';

end

\end{lstlisting}
 
\section{Modelo lineal discretizado}

%\newpage
\begin{lstlisting}[frame=single]
function [Ad,Bd,Cd] = get_discrete_linear_model(A,B,C,Ts)
%UNTITLED6 Summary of this function goes here
%   Detailed explanation goes here

sys = ss(A,B,C,0); 
sys_d  = c2d(sys,Ts);
Ad = sys_d.A;
Bd = sys_d.B;
Cd = sys_d.C;
end


\end{lstlisting}
 
\section{Modelo no lineal}

%\newpage
\begin{lstlisting}[frame=single]
function [] = generate_non_linear_model()
%UNTITLED5 Summary of this function goes here
%   Detailed explanation goes here

[xp,y] = get_symbolic_model('simulation');

fileID = fopen('hmb_non_linear_model.m','w');
head = 'function [xp,y] =
hmb_non_linear_model(x,u,ws,Fd_x,Fd_y,Fd_z,Md_y,Md_z,Fmcce1,Fmcce2)';
tail  = 'end';

fprintf(fileID,'%s\n',head); %% function header
%%
fprintf(fileID,'\n[m,R_r,R_a,R_p,I_p,I_0] = 
system_parameters();\n');
%% assign input variables
fprintf(fileID,'\n');
for r = 1:size(xp,1)
    fprintf(fileID,'x%d = x(%d);\n',r,r);
end

fprintf(fileID,'\n');

for r = 1:5
    fprintf(fileID,'u%d = u(%d);\n',r,r);
end
%% write xp
fprintf(fileID,'\n');
fprintf(fileID,'%s\n','xp = [');
for r = 1:size(xp,1)
    fprintf(fileID,'\t%s;\n',char(xp(r)));    
end
fprintf(fileID,'%s\n','];');

%% write y
fprintf(fileID,'\n');
fprintf(fileID,'%s\n','y = [');
for r = 1:size(y,1)
    fprintf(fileID,'\t%s;\n',char(y(r)));
end
fprintf(fileID,'%s\n','];');
%% function tail
fprintf(fileID,'\n');
fprintf(fileID,'%s\n',tail);
fclose(fileID);

end


\end{lstlisting}

\section{Ganancias de control} \label{ap:ganancias-control}

%\newpage
\begin{lstlisting}[frame=single]
function [K,Ko] = calc_linear_controller_gains(A,B,C,Pc,Po)
%	UNTITLED2 Summary of this function goes here
%   Detailed explanation goes here
K = place(A,B,Pc);
Ko = place(A',C',Po)';
end


\end{lstlisting}

\section{Ganancias de control integral} \label{ap:ganancias-control-integral}

%\newpage
\begin{lstlisting}[frame=single]
function [K,Ai,Bi,Ci] = calc_integral_controller_gains(A,B,C,P)
%	UNTITLED6 Summary of this function goes here
%   Detailed explanation goes here

m = size(B,2);% inputs
n = size(B,1);% states
p =  size(C,1);% outputs

Ai =[A zeros(n,p);
      -C zeros(p,p)];
Bi = [B;zeros(p,m)];

Ci = [C zeros(p,p)];

K = place(Ai,Bi,P);
end


\end{lstlisting}

\section{Ganancias de control integral discreto}

%\newpage
\begin{lstlisting}[frame=single]
function [Kdi,Adi,Bdi] = 
		calc_discrete_integral_controller_gains(A,B,C,Pi,Ts)
%	UNTITLED10 Summary of this function goes here
%   Detailed explanation goes here
[Ad,Bd,Cd] = get_discrete_linear_model(A,B,C,Ts);
Pdi = get_discrete_poles(Pi,Ts);

m = size(B,2);% inputs
n = size(B,1);% states
p =  size(C,1);% outputs

Adi =[Ad zeros(n,p);
      -Cd eye(p,p)];
Bdi = [Bd;zeros(p,m)];

Kdi = place(Adi,Bdi,Pdi);
end

\end{lstlisting}

\section{Ganancias de control  discreto}

%\newpage
\begin{lstlisting}[frame=single]
function [Kd,Kdo] = calc_discrete_controller_gains(A,B,C,P,Po,Ts)
%UNTITLED7 Summary of this function goes here
%   Detailed explanation goes here
[Ad,Bd,Cd] = get_discrete_linear_model(A,B,C,Ts);
Pd = get_discrete_poles(P,Ts);
Pdo = get_discrete_poles(Po,Ts);
[Kd,Kdo] = calc_linear_controller_gains(Ad,Bd,Cd,Pd,Pdo);
end

\end{lstlisting}

\section{Capacitancia}

%\newpage
\begin{lstlisting}[frame=single]

function [ capacitance, r_R ] = get_cap_vs_curv(y)
%GET_CAP_VS_CURV Summary of this function goes here
%   Detailed explanation goes here
y_0 = 1.0;

p1 = [0,y_0];
p2 = [10,y_0];

r_R = 0.1 : 0.1: 1;
r = 10;
R = r./r_R;
alpha_rad = asin(r_R);
alpha_deg = rad2deg(alpha_rad);
cos_alpha = cos(alpha_rad);
h = R.*(1-cos_alpha);

capacitance = [];

openfemm()
for n = 1:length(h)
opendocument('cds_param_sim.FEE');
ei_saveas('temp.FEE');

ei_clearselected();
ei_selectgroup(4)
ei_deleteselected();

ei_clearselected();
ei_selectnode(p2);
ei_movetranslate(0,h(n));

ei_addarc(p1(1),p1(2),p2(1),h(n)+p2(2),alpha_deg(n),1)

ei_clearselected();
ei_selectgroup(3)
ei_movetranslate(0,y-y_0);

ei_analyze(1);

ei_loadsolution();
conductor_properties = eo_getconductorproperties('sensor');
capacitance = [capacitance conductor_properties(2)];
eo_close()
ei_close()
end
closefemm();
%%
openfemm()
opendocument('curvature_simulation.FEE');
ei_saveas('temp.FEE');
ei_clearselected();
ei_selectgroup(3)
ei_movetranslate(0,y-y_0);
ei_analyze(1);
ei_loadsolution();
conductor_properties = eo_getconductorproperties('sensor');
capacitance = [conductor_properties(2) capacitance];
r_R = [0 r_R];
closefemm();
end
\end{lstlisting}

\section{Sensor de distancia capacitivo}

%\newpage
\begin{lstlisting}[frame=single]
function [ capacitance ] = cds_calc( y,d_i,r_R)
%UNTITLED Summary of this function goes here
%   Detailed explanation goes here
y_0 = 1.0;
d_0 = 1.0;

p1 = [0,y_0];
p2 = [10,y_0];

r = 10;
R = r./r_R;
alpha_rad = asin(r_R);
alpha_deg = rad2deg(alpha_rad);
cos_alpha = cos(alpha_rad);
h = R.*(1-cos_alpha);

opendocument('cds_param_sim.FEE');
ei_saveas('temp.FEE');

if(r_R ~= 0)
    ei_clearselected();
    ei_selectgroup(4)
    ei_deleteselected();

    ei_clearselected();
    ei_selectnode(p2);
    ei_movetranslate(0,h);

    ei_addarc(p1(1),p1(2),p2(1),h+p2(2),alpha_deg,1)
end

ei_clearselected();
ei_selectgroup(2)
ei_movetranslate(d_i-d_0,0);

ei_clearselected();
ei_selectgroup(3)
ei_movetranslate(0,y-y_0);

ei_analyze(1);

ei_loadsolution();
conductor_properties = eo_getconductorproperties('sensor');
capacitance = conductor_properties(2);

eo_close()
ei_close()
end

\end{lstlisting}

\section{Geometria del sensor}

%\newpage
\begin{lstlisting}[frame=single]
function [ capacitance, r] = simulate_geometry_at(r_max,delta_r,y_i)
%UNTITLED Summary of this function goes here
%   Detailed explanation goes here
r_min = 1;
r = r_min:delta_r:r_max;
%%
capacitance = [];
openfemm
opendocument('gp_simulation.FEE');
ei_saveas('temp.FEE');

ei_clearselected
ei_selectgroup(1);
ei_movetranslate(0,y_i-0.5);
%%
for r_loop = r
    
    ei_analyze(1)
    ei_loadsolution
    conductor_properties = eo_getconductorproperties('sensor');
    
    capacitance_loop = conductor_properties(2);
    capacitance = [capacitance capacitance_loop]; 
    
    ei_clearselected
    ei_selectgroup(2);
    ei_movetranslate(delta_r,0);
end
closefemm

end

\end{lstlisting}

\section{Curvatura y sensitividad}

%\newpage
\begin{lstlisting}[frame=single]
function [ delta_c ] = sensitivity_at_curvature(y_0,y_1,d_i, curvature)
%UNTITLED2 Summary of this function goes here
%   y_0 miminum distance
%   y1 maximum_distance
%   d_i vector of diameters
%   d_i vector of diameters
c_0 = [];
c_1 = [];

openfemm();
for d = d_i;
c_0 = [c_0 cds_calc(y_0, d, curvature)];
c_1 = [c_1 cds_calc(y_1, d, curvature)];
end
closefemm();

delta_c = c_1 - c_0;
end

\end{lstlisting}